% Options for packages loaded elsewhere
\PassOptionsToPackage{unicode}{hyperref}
\PassOptionsToPackage{hyphens}{url}
%
\documentclass[
]{article}
\usepackage{lmodern}
\usepackage{amssymb,amsmath}
\usepackage{ifxetex,ifluatex}
\ifnum 0\ifxetex 1\fi\ifluatex 1\fi=0 % if pdftex
  \usepackage[T1]{fontenc}
  \usepackage[utf8]{inputenc}
  \usepackage{textcomp} % provide euro and other symbols
\else % if luatex or xetex
  \usepackage{unicode-math}
  \defaultfontfeatures{Scale=MatchLowercase}
  \defaultfontfeatures[\rmfamily]{Ligatures=TeX,Scale=1}
\fi
% Use upquote if available, for straight quotes in verbatim environments
\IfFileExists{upquote.sty}{\usepackage{upquote}}{}
\IfFileExists{microtype.sty}{% use microtype if available
  \usepackage[]{microtype}
  \UseMicrotypeSet[protrusion]{basicmath} % disable protrusion for tt fonts
}{}
\makeatletter
\@ifundefined{KOMAClassName}{% if non-KOMA class
  \IfFileExists{parskip.sty}{%
    \usepackage{parskip}
  }{% else
    \setlength{\parindent}{0pt}
    \setlength{\parskip}{6pt plus 2pt minus 1pt}}
}{% if KOMA class
  \KOMAoptions{parskip=half}}
\makeatother
\usepackage{xcolor}
\IfFileExists{xurl.sty}{\usepackage{xurl}}{} % add URL line breaks if available
\IfFileExists{bookmark.sty}{\usepackage{bookmark}}{\usepackage{hyperref}}
\hypersetup{
  hidelinks,
  pdfcreator={LaTeX via pandoc}}
\urlstyle{same} % disable monospaced font for URLs
\setlength{\emergencystretch}{3em} % prevent overfull lines
\providecommand{\tightlist}{%
  \setlength{\itemsep}{0pt}\setlength{\parskip}{0pt}}
\setcounter{secnumdepth}{-\maxdimen} % remove section numbering

\author{}
\date{}

\begin{document}

\hypertarget{free-code-camp-course---linux-essentials-for-hackers---4-hours}{%
\section{Free Code Camp Course - Linux Essentials for Hackers - 4
hours}\label{free-code-camp-course---linux-essentials-for-hackers---4-hours}}

\begin{itemize}
\tightlist
\item
  linux is imoprtant for

  \begin{itemize}
  \tightlist
  \item
    security
  \item
    system administration
  \item
    personal use! (I want this :) )
  \end{itemize}
\item
  linux can run on

  \begin{itemize}
  \tightlist
  \item
    virtual machine
  \item
    baremetal
  \end{itemize}
\end{itemize}

\hypertarget{useful-keyboard-shortcuts}{%
\subsection{useful keyboard shortcuts}\label{useful-keyboard-shortcuts}}

\begin{itemize}
\tightlist
\item
  open terminal

  \begin{itemize}
  \tightlist
  \item
    \texttt{ctrl\ +\ alt\ +\ t}
  \end{itemize}
\item
  move window to top, right, bottom, left

  \begin{itemize}
  \tightlist
  \item
    \texttt{window\ +\ arrowkeys}
  \end{itemize}
\item
  increase/decrease font size

  \begin{itemize}
  \tightlist
  \item
    \texttt{ctrl\ +\ shift\ +\ "+"}
  \item
    \texttt{ctrl\ +\ "-"}
  \end{itemize}
\item
  clear the terminal

  \begin{itemize}
  \tightlist
  \item
    \texttt{ctrl\ +\ l}
  \item
    \texttt{clear} command
  \end{itemize}
\item
  end the current process

  \begin{itemize}
  \tightlist
  \item
    \texttt{ctrl\ +\ c}
  \end{itemize}
\item
  see history commands

  \begin{itemize}
  \tightlist
  \item
    \texttt{top\ and\ bottom\ in\ arrowkeys}
  \end{itemize}
\item
  auto complete command

  \begin{itemize}
  \tightlist
  \item
    \texttt{tab}
  \item
    \texttt{right-arrowkey}
  \end{itemize}
\item
  close the window

  \begin{itemize}
  \tightlist
  \item
    \texttt{ctrl\ +\ w}
  \item
    \texttt{ctrl\ +\ shift\ +\ w}
  \end{itemize}
\end{itemize}

\hypertarget{file-management-and-manipulation}{%
\subsection{file management and
manipulation}\label{file-management-and-manipulation}}

\begin{itemize}
\item
  print working directory

\begin{verbatim}
pwd
\end{verbatim}
\item
  list directory

\begin{verbatim}
ls
\end{verbatim}
\item
  list direcotry in a table

\begin{verbatim}
ls -l
\end{verbatim}
\item
  list directory also hiddens

\begin{verbatim}
ls -a
\end{verbatim}
\item
  list direcotry in a table human readable

\begin{verbatim}
ls -lh
\end{verbatim}
\item
  list direcotry recursively!

\begin{verbatim}
ls -R
\end{verbatim}
\item
  cahnge directory to home directory

\begin{verbatim}
cd
\end{verbatim}

\begin{verbatim}
cd ~
\end{verbatim}
\item
  change directory to previous directory

\begin{verbatim}
cd -
\end{verbatim}
\item
  change directory to parent directory

\begin{verbatim}
cd ..
\end{verbatim}
\item
  change directory to any directory you want

\begin{verbatim}
cd wanteddirectorypath
\end{verbatim}
\item
  change directory to root directory

\begin{verbatim}
cd /
\end{verbatim}
\item
  see the one line documentation of a command!!!

\begin{verbatim}
whatis thecommand
\end{verbatim}
\item
  create new file

\begin{verbatim}
touch newfilename
\end{verbatim}
\item
  returning a line

\begin{verbatim}
echo yourlinegoeshere
\end{verbatim}
\item
  redirect output to a file

\begin{verbatim}
echo "something" > somefile
\end{verbatim}
\item
  see the content of a file

\begin{verbatim}
cat somefile
\end{verbatim}
\item
  copy content of a file to another file

\begin{verbatim}
cat somefile > anotherfile
\end{verbatim}
\item
  remove a file or directory

\begin{verbatim}
rm yourfile
\end{verbatim}

\begin{verbatim}
rm -r yourdirectory
\end{verbatim}
\item
  remove all files and folders in a directory

\begin{verbatim}
rm -r *
\end{verbatim}
\item
  create a directory

\begin{verbatim}
mkdir yournewdirectoryname
\end{verbatim}
\item
  copy a file to another directory

\begin{verbatim}
cp yourfilepath yournewdirectory
\end{verbatim}
\item
  copy a directory to another directory

\begin{verbatim}
cp -r yourfolderpath yournewdirectory
\end{verbatim}
\item
  move a file or folder to new directory

\begin{verbatim}
mv yourfileorfolderpath yournewdirectory
\end{verbatim}
\item
  rename a file

\begin{verbatim}
mv yourfileorfolderpath yournewname
\end{verbatim}
\item
  remove a direcotry

\begin{verbatim}
rmdir yourdirectorypath
\end{verbatim}
\item
  open a file with nano editor

\begin{verbatim}
nano filepath
\end{verbatim}
\item
  open a file with vim editor

\begin{verbatim}
vi filepath
\end{verbatim}
\end{itemize}

\hypertarget{file-and-directory-permissions}{%
\subsection{file and directory
permissions}\label{file-and-directory-permissions}}

\begin{itemize}
\tightlist
\item
  in a file config: \texttt{drwxrwxrwx}

  \begin{itemize}
  \tightlist
  \item
    \texttt{d} shows directory, \texttt{-} shows file
  \item
    \texttt{r} shows read permission
  \item
    \texttt{w} shows write permission
  \item
    \texttt{x} shows execute permission
  \item
    \texttt{-} shows without permission
  \item
    first \texttt{rwx} is for owner of file
  \item
    second \texttt{rwx} is for group of file
  \item
    third \texttt{rwx} is for others of file
  \end{itemize}
\item
  change the permission of file
\end{itemize}

\begin{verbatim}
chmod mapuserstopermissions filepath
\end{verbatim}

\begin{itemize}
\item
  map users to permissions

  \begin{itemize}
  \tightlist
  \item
    \texttt{ugo}: owner, group, others
  \item
    \texttt{rwx}: read, write, execute
  \item
    \texttt{=+-}: equal, append, delete
  \end{itemize}
\item
  also you can use from binary to map users to permissions
\item
  change the permission of folder
\end{itemize}

\begin{verbatim}
chmod -R mapuserstopermissions folderpath
\end{verbatim}

\hypertarget{file-and-directory-ownership}{%
\subsection{file and directory
ownership}\label{file-and-directory-ownership}}

\begin{itemize}
\item
  every file has a user and also a group
\item
  change the owner of file
\end{itemize}

\begin{verbatim}
chown newowner filepath
\end{verbatim}

\begin{itemize}
\tightlist
\item
  change the group of file
\end{itemize}

\begin{verbatim}
chgrp newgroup filepath
\end{verbatim}

\begin{itemize}
\tightlist
\item
  see the groups
\end{itemize}

\begin{verbatim}
groups
\end{verbatim}

\begin{itemize}
\tightlist
\item
  see the groups of specific user
\end{itemize}

\begin{verbatim}
groups username
\end{verbatim}

\begin{itemize}
\tightlist
\item
  see the current logined users
\end{itemize}

\begin{verbatim}
users
\end{verbatim}

\hypertarget{grep-and-piping}{%
\subsection{\texorpdfstring{\texttt{grep} and
piping}{grep and piping}}\label{grep-and-piping}}

\begin{verbatim}
$ whatis grep
grep (1)             - print lines that match patterns
\end{verbatim}

\begin{itemize}
\tightlist
\item
  usually \texttt{grep} used in 2 ways

  \begin{itemize}
  \tightlist
  \item
    direct
  \item
    pipe the last command
  \end{itemize}
\end{itemize}

\hypertarget{direct-grep}{%
\subsubsection{\texorpdfstring{direct
\texttt{grep}}{direct grep}}\label{direct-grep}}

\begin{itemize}
\tightlist
\item
  searching in file
\end{itemize}

\begin{verbatim}
grep "word" filepath
\end{verbatim}

\begin{quote}
\texttt{-i} usually used for case-insensitive way
\end{quote}

\hypertarget{grep-with-pipe}{%
\subsubsection{\texorpdfstring{\texttt{grep} with
pipe}{grep with pipe}}\label{grep-with-pipe}}

\begin{itemize}
\tightlist
\item
  simple use
\end{itemize}

\begin{verbatim}
yourfirstcommandwithoutput | grep "wordsearchinginoutput"
\end{verbatim}

\hypertarget{finding-files-with-locate}{%
\subsection{\texorpdfstring{finding files with
\texttt{locate}}{finding files with locate}}\label{finding-files-with-locate}}

\begin{verbatim}
$ whatis locate
locate (1)           - find files by name, quickly
\end{verbatim}

\begin{itemize}
\tightlist
\item
  actually \texttt{locate} is not very practical command in my
  opinion\ldots{}
\end{itemize}

\hypertarget{enumerating-distribution-and-kernel-information}{%
\subsection{enumerating distribution and kernel
information}\label{enumerating-distribution-and-kernel-information}}

\begin{itemize}
\tightlist
\item
  current user
\end{itemize}

\begin{verbatim}
whoami
\end{verbatim}

\begin{itemize}
\tightlist
\item
  os(workstation) name
\end{itemize}

\begin{verbatim}
hostname
\end{verbatim}

\begin{itemize}
\tightlist
\item
  change \texttt{hostname}
\end{itemize}

\begin{verbatim}
sudo nano /etc/hostname
\end{verbatim}

\begin{itemize}
\tightlist
\item
  see the linux distribution
\end{itemize}

\begin{verbatim}
lsb_release -a
\end{verbatim}

\begin{verbatim}
cat /etc/issue
\end{verbatim}

\begin{verbatim}
cat /etc/os-release
\end{verbatim}

\begin{verbatim}
cat /etc/*release
\end{verbatim}

\begin{verbatim}
uname -a
\end{verbatim}

\begin{itemize}
\tightlist
\item
  cpu information
\end{itemize}

\begin{verbatim}
lscpu
\end{verbatim}

\begin{itemize}
\tightlist
\item
  pci information
\end{itemize}

\begin{verbatim}
lspci
\end{verbatim}

\hypertarget{find-and-bandit-challanges}{%
\subsection{find and bandit
challanges}\label{find-and-bandit-challanges}}

\begin{verbatim}
$ whatis find    
find (1)             - search for files(also directories) in a directory hierarchy
\end{verbatim}

\begin{itemize}
\item
  very powerful command!
\item
  usage
\end{itemize}

\begin{verbatim}
find pathtosearch specifiers
\end{verbatim}

\begin{itemize}
\tightlist
\item
  specifiers

  \begin{itemize}
  \tightlist
  \item
    \texttt{-type}
  \item
    \texttt{-name}
  \item
    \texttt{-iname}
  \item
    \texttt{-size}
  \item
    \texttt{-perm}
  \end{itemize}
\end{itemize}

\end{document}
